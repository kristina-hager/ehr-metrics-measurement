\documentclass[10pt]{article} 
\PassOptionsToPackage{hyphens}{url}
\usepackage {xcolor}
\usepackage{hyperref} 
\hypersetup{
  colorlinks=true,
  linkcolor=blue,
  citecolor=blue,
  urlcolor=blue,
  linkbordercolor=blue
}
\usepackage{simpleConference}
\usepackage{times}
\usepackage{graphicx}
\graphicspath{ {images/} }
\usepackage{amssymb}
\usepackage{enumitem}
\setlist[description]{leftmargin=\parindent,labelindent=\parindent}
\begin{document}

\title{My EHR Paper title}

\author{Kristina Hager\\
Software Metrics and Measurement EE382V\\
University of Texas at Austin\\
April 30, 2015\\
}

\maketitle
\thispagestyle{empty}

\begin{abstract}
	This is an abstract!
\end{abstract}

% \tableofcontents

\section{Introduction}
\label{Introduction}

Electronic health record (EHR) systems collect health information about individual patients 
and enable exchange of a patient's information between healthcare providers.
These systems and the general computerization of healthcare carry a promise to today's healthcare consumers
of revolutionizing the quality, ease and cost of care.
They also present a great opportunity for software organizations to benefit society in meaningful ways
and also to generate a profit.

An EHR system must deliver the required, and legislated, functionality to enable improvements in delivery of care.
Since in the United States a specific EHR solution is not mandated, providers of EHR systems must compete in the marketplace
in order to win contracts with medical providers.
Althogh recent legislation in the United States has provided financial incentives to providers to ease the cost of adopting EHR,
the incentives are not enough to cover the full cost.
Medical providers, like everyone, are thus sensitive to the cost, effort, and maintanance required by an EHR system.
Therefore, a cloud-based EHR service is very alluring to providers due to reduced infrastructure costs typically required by cloud solutions \cite{auditingprivacy}.
% TODO -- strengthen argument here regarding legislation that mandates the functionality
% and that this paper won't focus on functionality since this is mandated
% but rather on the non-functional requirements that give a product it's edge
% 

In this paper, I will discuss cloud-based Software-as-a-Service (SaaS) Electronic Health Record (EHR) systems (EHR-SaaS) for the perspective
of the software community development community and a medical practice administrator in charge of selecting such a system.

My goal is to familiarize the reader with the goals of the EHR-SaaS product and the qualities important in an EHR-SaaS product
she may be developing or considering for purchase.
From these qualities, I will suggest a path down the Goals-Question-Metrics (GQM) methodology that can produce metrics relevant for an EHR-SaaS product.
TODO: doing this is TBD. may also get into web metrics.

In section \ref{sec:What is an EHR}, I will present an overview of the definition and goals of an EHR.

In section TODO, I will present an overview of non-functional requirements that are critical to a SaaS product.

In section TODO, I will present the non-functional requirements I suspect are important for an EHR-SaaS product.

In section TODO, I will cover la-tee-daTODO.

\section{What is an EHR?}
\label{sec:What is an EHR}

An EHR is most easily explained as a digital version of a patient's paper chart.
However, an EHR exhbits features well beyond that of an ordinary paper chart.
It will contain real time information from multiple providers from different organizations and be able to be securely and instantaneously shared between authorized users.
Specifically, an EHR will contain a patient's medical history, previous and current diagnoses, medications, treatment plans, immunization dates, radiology images, laboratory results, and test results.

An EHR system contains these digital patient histories and enables further sophisticated functions.
For example, an EHR system gives providers access to evidence-based tools to use when making decisions about patient care.
The system also automates and streamlines the provider's workflow, i.e. referrals and patient records can be automatically transferred between cooperating providers.
\cite{healthit-ehr}

Sinclair, et. al. state in \cite{auditingprivacy}, that "An EHR is a modern specialization of a CRM (customer relationship management) which specifically focuses on the collection and exchange of electronic health information about individual patients between healthcare organizations".


\subsection{Goals of an EHR}
\label{sec:Goals of an EHR}

The goals of an EHR are oriented around improving the care the patient receives, assuring security and privacy, and enabling further work in evidence-based medicine and in public health objectives: 
\begin{description}
\item[Continuity of Care] - The provision of a comprehensive record of care for the lifetime of every patient and thus a higher level of ongoing healthcare quality
\item[Access and Security] - A means of properly controlling and providing access; standardization and organization
\item[Reduction of Medical Errors] - Reduces errors by eliminating handwritten orders which can be misinterpreted; enable drug-drug interaction validation and cross-referencing of drug allergies and the patient's vital statistics
\item[Increased Patient Access and Awareness] - Patient access portals enable a patient to be aware of and connected to their medical record, their treatments and results
\item[Evidence-Based Medicine] - Collection of patient medical data, treatments and results into connected databases enabled unprecedented research over long period of time
\item[Public Health Reporting] - The collection of patient data as above supports and improves efficiency in public health efforts in areas like tracking and researching communicable illnesses and chronic conditions
\end{description}
% TODO - indent this table

These goals are outlined in many reputable resources. I have drawn my information from \cite{ehrbook}.

\subsection{Non-Functional Requirements of an EHR}
\label{sec:NFR of an EHR}

\section{Related Work}
\label{sec:Related}
etc

\section{Conclusions and Future Work}

\subsection{Conclusions-Subsection}
\label{sec:Conclusions-Subsection}

Conclusions go here. 

\subsection{Future Work}
\label{sec:Future Work}
Thoughts on future work go here.

\bibliographystyle{abbrv}
\bibliography{refs}
\end{document}
