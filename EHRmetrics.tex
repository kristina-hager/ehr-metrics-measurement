\documentclass[10pt]{article} 
\PassOptionsToPackage{hyphens}{url}
\usepackage {xcolor}
\usepackage{hyperref} 
\hypersetup{
  colorlinks=true,
  linkcolor=blue,
  citecolor=blue,
  urlcolor=blue,
  linkbordercolor=blue
}
\usepackage{simpleConference}
\usepackage{times}
\usepackage{graphicx}
\graphicspath{ {images/} }
\usepackage{amssymb}
\begin{document}

\title{My EHR Paper title}

\author{Kristina Hager\\
Software Metrics and Measurement EE382V\\
University of Texas at Austin\\
April 30, 2015\\
}

\maketitle
\thispagestyle{empty}

\begin{abstract}
	This is an abstract!
\end{abstract}

% \tableofcontents

\section{Introduction}
\label{Introduction}

Electronic health record (EHR) systems collect health information about individual patients 
and enable exchange of a patient's information between healthcare providers.
These systems and the general computerization of healthcare carry a promise to today's healthcare consumers
of revolutionizing the quality, ease and cost of care.
They also present a great opportunity for software organizations to benefit society in meaningful ways
and also to generate a profit.

An EHR system must deliver the required, and legislated, functionality to enable improvements in delivery of care.
Since in the United States a specific EHR solution is not mandated, providers of EHR systems must compete in the marketplace
in order to win contracts with medical providers.
Althogh recent legislation in the United States has provided financial incentives to providers to ease the cost of adopting EHR,
the incentives are not enough to cover the full cost.
Medical providers, like everyone, are thus sensitive to the cost, effort, and maintanance required by an EHR system.
Therefore, a cloud-based EHR service is very alluring to providers due to reduced infrastructure costs typically required by cloud solutions \cite{auditingprivacy}.

In this paper, I will discuss cloud-based Software-as-a-Service (SaaS) Electronic Health Record (EHR) systems (EHR-SaaS) for the perspective
of the software community development community and a medical practice administrator in charge of selecting such a system.

My goal is to familiarize the reader with the goals of the EHR-SaaS product and the qualities important in an EHR-SaaS product
she may be developing or considering for purchase.
From these qualities, I will suggest a path down the Goals-Question-Metrics (GQM) methodology that can produce metrics relevant for an EHR-SaaS product.
TODO: doing this is TBD. may also get into web metrics.

In section \ref{sec:What is an EHR}, I will present an overview of the definition and goals of an EHR.

In section TODO, I will present an overview of non-functional requirements that are critical to a SaaS product.

In section TODO, I will present the non-functional requirements I suspect are important for an EHR-SaaS product.

In section TODO, I will cover la-tee-daTODO.

\section{What is an EHR?}
\label{sec:What is an EHR}
Describe an EHR here \cite{auditingprivacy}


\subsection{Nonfunctional Requirements of an EHR}
\label{sec:Nonfunctional Requirements of an EHR}
I will outline NFR of EHR here.

\section{Related Work}
\label{sec:Related}
etc

\section{Conclusions and Future Work}

\subsection{Conclusions-Subsection}
\label{sec:Conclusions-Subsection}

Conclusions go here. 

\subsection{Future Work}
\label{sec:Future Work}
Thoughts on future work go here.

\bibliographystyle{abbrv}
\bibliography{refs}
\end{document}
