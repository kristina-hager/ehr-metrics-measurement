\documentclass[10pt]{article} 
\PassOptionsToPackage{hyphens}{url}
\usepackage {xcolor}
\usepackage{hyperref} 
\hypersetup{
  colorlinks=true,
  linkcolor=blue,
  citecolor=blue,
  urlcolor=blue,
  linkbordercolor=blue
}
\usepackage{simpleConference}
\usepackage{times}
\usepackage{graphicx}
\graphicspath{ {images/} }
\usepackage{amssymb}
\usepackage{enumitem}
\setlist[description]{leftmargin=\parindent,labelindent=\parindent}
\begin{document}

%\title{My EHR Paper title}

%\author{Kristina Hager\\
%Software Metrics and Measurement EE382V\\
%University of Texas at Austin\\
%April 30, 2015\\
%}

%\maketitle
%\thispagestyle{empty}

%\begin{abstract}
%	This is an abstract!
%\end{abstract}

% \tableofcontents

\section{Introduction}
\label{sec:Introduction}

Electronic health record (EHR) systems are computerized systems that collect health information about individual patients 
and enable exchange of a patient's information between healthcare providers.
EHR and the overall computerization of healthcare promises to revolutionize the quality, ease and cost of care for consumers, payers and providers of healthcare alike.
These systems also present a not only a great new market opportunity for software organizations and also a way software organizations can benefit society in meaningful ways.

% Specific legislation in the United States has paved the way and, in essence, mandated that providers adopt EHR systems.
% PHI definition: https://www.hipaa.com/hipaa-protected-health-information-what-does-phi-include/

The Health Insurance Portability and Accountability Act (HIPAA) of 1996 laid essential groundwork for EHR systems by defining and regulating the exchange of Protected Health Information (PHI) and by establishing certain national standards for healthcare data and exchange.
PHI can be found in myriad places. 
It is essentially any information that can positively identify a person and/or connect them to any health care related information.
HIPAA provides protection for PHI and specifies severe criminal and civil penalties for mishandling of this data.
%Therefore, any EHR system must provide for protection of PHI in compliance with this legislation.
HIPAA also established standards regarding the processing of claims, benefits, pharmacy transactions, and code sets for medical procedures and diagnoses. 
\cite{ehrbook}

The Health Information Technology for Economic and Clinical Health (HITECH) Act of 2009 provides significant financial incentives for the adoption and \textit{Meaningful Use} (MU) of EHR systems \cite{ehrbook}.
Furthermore, the HITECH Act empowers the Office of the National Health Information Technology Coordinator (ONC) to define requirements for and certify EHR systems \cite{onc-ehr}.
HITECH also clearly specifies how providers can effectively use (as defined in Meaningful Use) these systems to improve quality and efficiency of care and to earn monetary incentives \cite{ehrbook}.

%An EHR system must be certified by the ONC as having the required functionality in order for the adopting provider to be eligible for incentives.
Medical providers can choose among certified EHR systems and select the system that they feel meets their needs and offers the most attractive return on investment.
Therefore, providers of EHR systems must compete with other vendors in the marketplace in order to win contracts with medical providers.
Since all EHR systems must achieve the regulated set of functionality to be eligible, vendors must necessarily concentrate on differentiating their systems with regards to non-functional requirements (quality factors) and price.
A cloud-based EHR service is very alluring to providers due to reduced infrastructure costs typically required by cloud solutions \cite{auditingprivacy}.
I will cover in this paper the quality factors that are likely to be important to purchasers of a cloud-based EHR system.

This paper is written primarily for a reader from the software industry with an interest in EHR, and it may also be of use to a medical provider administrator in charge of selecting such a system.
This paper can be divided into two major sections:
\begin{itemize}
	\item An overview of the high level goals of an EHR system, Meaningful Use, and what sorts of metrics the medical provider must produce, using an EHR system, to attest to Meaningful Use.
	\item An overview of SaaS, considerations specific to EHR-SaaS and suggest metrics an EHR-SaaS software vendor may use to differentiate its product.
\end{itemize}

In section \ref{sec:Intro_EHR}, I will introduce the EHR system.

In section \ref{sec:Goals of an EHR}, I will present a high level overview of the goals and non-functional requirements of such a system.

In section \ref{sec:Define Meaningful Use}, I will introduce the concept of Meaningful Use for the reader.

In section \ref{sec:Meaningful Use Metrics}, I will provide to the reader details and some examples of the metrics a provider must produce in order to attest to Meaningful Use and earn incentives.

In section \ref{sec:CQM}, I will provide an overview of the Clinical Quality Measures (CQM) that a provider must report to maintain Meaningful Use and earn incentives.

In section \ref{sec:Intro_SaaS}, I will introduce the concept of Software-as-a-Service (SaaS) and in describe the advantages and challenges of a SaaS solution for the EHR product in sections \ref{sec:SaaS Adantages} and \ref{sec:SaaS Challenges}.
 
In section \ref{sec:Critical-EhrSaaS-Ilities}, I will offer my thoughts how the providers of an EHR-SaaS system can differentiate their product in the marketplace by presenting the critical non-functional requirements of a generic SaaS product and my thoughts on the critical non-functional requirements for an EHR-SaaS product in sections \ref{sec:Critical-SaaS-Ilities} and \ref{sec:Critical-EHR-Ilities}. 
%I will first cover existing research by Benlian, Koufaris, and Hess (2011, \cite{saasqual}) on the critical non-functional requirements of a generic business-to-business SaaS product.
%With Benlian, et al.'s work as a solid foundation, I will offer a critique of their conclusions from the perspective of an EHR-SaaS system and a modified list of quality factors I conjecture to be critical for this unique product.

\section{Introducing the EHR System}
\label{sec:Intro_EHR}

An EHR is most concisely explained as a digital version of a patient's paper chart.
However, an individual EHR and an EHR system exhibits features well beyond that of an ordinary paper chart or set of paper charts.
An EHR will contain real time information from multiple providers from different organizations and can be securely and instantaneously shared between authorized users.
Specifically, an EHR will contain a patient's entire medical history, previous and current diagnoses, medications, treatment plans, immunization dates, radiology images, laboratory results, and test results from many, and ideally all, providers.

An EHR system contains these digital patient charts and enables further sophisticated functions.
For example, an EHR system gives providers access to evidence-based tools to use when making decisions about patient care.
The system also automates and streamlines the provider's workflow. For example, referrals and patient records can be automatically transferred between cooperating providers.
\cite{healthit-ehr}

Sinclair, Hudzia, and Stewart (2014) place EHR systems in the context of other software systems with their observation: "An EHR is a modern specialization of a CRM (customer relationship management) which specifically focuses on the collection and exchange of electronic health information about individual patients between healthcare organizations" \cite{auditingprivacy}.


\subsection{Goals and Non-Functional Requirements of an EHR System}
\label{sec:Goals of an EHR}

As mentioned in section \ref{sec:Introduction} the legislators of the United States have enacted, over the last couple of decades, large and wide reaching acts that effectively mandate the adoption of EHR systems by nearly all medical practices.
A discussion of why an act of Congress was required to encourage the medical industry to computerize, when most other industries computerized long ago, is extremely interesting but outside the scope of this paper.
However, suffice it to say, the legislators enacted these rules with the goals of improving the quality and efficiency of care for all users of the healthcare system and not merely for the benefit of software companies.
%These acts also enabled regulatory bodies to specify very clearly what an EHR system is, what standards it must obey, and how a practice must use these systems to achieve improvements in care.

The impetus for an EHR are oriented around improving the care the patient receives, assuring security and privacy, and enabling further work in evidence-based medicine and in public health objectives.
I have drawn this summary from \cite{ehrbook}, and the goals of EHR are as follows:
\begin{description}
	\item[Continuity of Care] The provision of a comprehensive record of care for the lifetime of every patient and thus a higher level of ongoing healthcare quality
	\item[Access and Security] Properly controlling and providing access to patient records; standardization and organization
	\item[Reduction of Medical Errors] Reduce errors by eliminating handwritten orders which can be misinterpreted; enable drug-drug interaction validation and cross-referencing of drug allergies and the patient's vital statistics
	\item[Increased Patient Access and Awareness] Patient access portals enable a patient to be aware of and connected to their medical record, their treatments and results
	\item[Evidence-Based Medicine] Collection of patient medical data, treatments and results into connected databases enabled unprecedented research over long period of time
	\item[Public Health Reporting] The collection of patient data as above supports and improves efficiency in public health efforts in areas like tracking and researching communicable illnesses and chronic conditions
\end{description}

%\subsection{Non-Functional Requirements of an EHR System}
%\label{sec:NFR of an EHR}

%As mentioned in section \ref{sec:Intro_EHR}, an EHR system product must differentiate itself primarily in the area of non-functionality requirements.
%In my research, the non-functional requirements that are mentioned repeatedly when EHR systems are described for the medical community are:

%In order to achieve the patient, research and public health goals of EHR, and to address previous issues in the execution of healthcare - whether via paper patient chart or whether in the early attempts at EHR systems in early adopters - the EHR system is desired to have the following non-functional attributes:

Sources covering the goals of EHR systems typically mentioned the following non-functional requirements alongside functional goals:

\begin{description}
	\item[Portability] (of the individual EHR) - ability to accurately and securely transfer the EHR of a patient from once provider to another \cite{ehrbook}
	\item[Security and Privacy] - the data contained in each message is readable only by intended and authorized recipient \cite{ehrbook}, \cite{auditingprivacy}
	\item[Interoperability] - connectivity between different provider organizations and EHR systems \cite{ehrbook}
	\item[Reliability] - orders and records remain in appropriate state and order even in the face of planned or unplanned system downtime  \cite{ehrbook}
\end{description}



% TODO - what about usability here? - not really mentioned in books, but I can cover this later
% TODO -- also beef up the intro to this by reivisiting sources
% also -- consider length of committment to EHR ehr-breakup if appropriate later

\subsection{What is Meaningful Use?}
\label{sec:Define Meaningful Use}

The phrase "Meaningful Use" was first mentioned in the HITECH Act of 2009.
This act provided for about \$20 billion, a significant sum, in funds to go to medical providers who "meaningfully use" EHR software to improve clinical care.
These incentives will help offset the cost of EHR software \cite{health-hack}.

Until the precise definition of "Meaningful Use" and associated certification requirements for EHR systems, there was no clear definition of what an EHR system actually needed to do.
Therefore, when the medical community asked for an EHR, they could not precisely and consistently specify the desired functionality.
As Dr. Ignacio Valdes, a highly regarded physician in the Health IT community, observed: "For decades, doctors had no idea what they wanted, and software developers gave it to them" \cite{health-hack}.

The definition of meaningful use criteria is established by about 5,000+ pages of legislation and rules from the ONC, United States Department of Health and Human Services (HHS), and Center for Medicare and Medicaid Services (CMS) and vary considerably based on whether the provider is am ambulatory (outpatient) practice or an inpatient (hospital) facility.
THe great benefit of the Meaningful Use guidelines is that they clearly specify what medical providers need to do to improve care quality and how they should use their EHR systems to do so.
It follows then that these guidelines clarify the functionality an EHR system needs to have.
%Trotter and Ulman (2012) in \underline{Hacking Healthcare} \cite{health-hack}, break the meaningful use objectives down into two distinct groups: a core set of (required) objectives and a "menu" set of objectives from which providers can choose to implement 5 of 10 requirements.

\subsection{Example Metrics Required for Meaningful Use}
\label{sec:Meaningful Use Metrics}

CMS breaks meaningful use into Stage 1, 2 and 3 attestation levels.
With each stage, a facility must meet a prescribed set of objectives and choose a handful of objectives to meet among a specified menu \cite{cms-stage1}.
The objectives are separated into sets for inpatient facilities and outpatient facilities.
In this paper, I will discuss the specifics for outpatient facilities for the sake of simplicity and brevity.
The inpatient facility objectives are in the same spirit as for outpatient facilities, but modified for the particular care model of hospitals, i.e. briefly treating acute conditions, versus an outpatient facility which generally handles long term, preventative care.
Furthermore, the CMS requires that each facility continue to submit Clinical Quality Measures (CQM) (see section \ref{sec:CQM}) for each year they participate in the Meaningful Use program.

Each objective in Meaningful Use includes a statement of the objective, a specific measure for meeting the objective, and, when applicable, a specific exclusion standard from the objective.
Many of the objective measures require a certain percentage of patients or orders to be handled with the EHR system to meet the objective.
Other objective measures are simply a yes/no answer to whether or not a certain objective was implemented.
Some objectives require a litmus test that the facility can perform a certain functionality. Examples of these questions are listed in table \ref{table:mu_metrics_example}.

\begin{table}
	\begin{tabular}{ |p{17.5cm}| }
		\hline
		\textbf{Example 1 - Percentage (Core Measure 1)} \\
		\hline
		\textbf{Objective} \\
		Use computerized provider order entry (CPOE) for medication orders directly entered by a licensed healthcare professional who can enter orders into the medical record per state, local and professional guidelines. \\
		\textbf{Measure} \\
		More than 30 percent of all unique patients with at least one medication in their medication list seen by the Eligible Professional (EP) have at least one medication order entered using CPOE. \\
		\textbf{Exclusion} \\
		Based on ALL patient records: Any EP [eligible provider] who writes fewer than 100 prescriptions during the EHR reporting period would be excluded from this requirement. Exclusion from  this requirement does not prevent an EP from achieving meaningful use. \\
		\hline
		\textbf{Example 2 - Yes/No (Core Measure 10)} \\
		\hline
		\textbf{Objective} \\
		Implement one clinical decision support rule relevant to specialty or high clinical priority along with the ability to track compliance to that rule. \\
		\textbf{Measure} \\
		Have you implemented one clinical decision support rule relevant to specialty or high clinical priority along with the ability to track compliance to that rule? \\
		\hline
		\textbf{Example 3 - Proof of Implementation (Menu Measure 1)} \\
		\hline
		\textbf{Objective} \\
		Capability to submit electronic data to immunization registries or Immunization information systems and actual submission according to applicable law and practice. \\
		\textbf{Measure} \\
		Did you perform at least one test of certified EHR technology’s capacity to submit electronic data to immunization registries and follow up submission if the test is successful (unless none of the immunization registries to which the eligible professional submits such information has the capacity to receive the information electronically), except where prohibited? \\
		\hline
		\end{tabular}
		\label{table:mu_metrics_example}
		\caption[Table caption text]{Example meaningful use objectives and measures taken from \cite{mu1-calc}}
\end{table}


Based on my review of the Stage 1 and 2 Meaningful Use objectives, I would categorize the objectives in the support of at least the following high level goals and categories:
\begin{itemize}
	\item Reduce errors by electronically transmitting information (prescrptions, etc)
	\item Submit data electronically to public health bureaucracies (immunizations, disease info)
	\item Communicate electronically between provider organizations
	\item Enable research or targeted patient outreach
	\item Improve communication with patient
	\item Improve follow up with patient about specific conditions
	\item Improve transition of patient's care between providers
	\item Improve patient safety via computer checking of allergies, medication conflicts, and computerized decision support
	\item Ensure privacy and security of patient information (PHI)
	\item Store patient data as structured data wherever possible (i.e. diagnoses, medications, demographic info)
\end{itemize}

\subsection{Reporting Clinical Quality Measures}
\label{sec:CQM}

\begin{table}
	\begin{tabular}{ |p{5cm}|p{12.5cm}| }
		\hline
		\textbf{Title} & \textbf{Description of Measure} \\
		\hline
		Hypertension: Blood Pressure Measurement & Percentage of patient visits for patients aged 18 years and older with a diagnosis of hypertension who has been seen for at least 2 office visits, with blood pressure (BP) recorded. \\
		\hline
		Preventive Care and Screening: Influenza Immunization for Patients ≥ 50 Years Old & Percentage of patients aged 50 years and older who received an influenza immunization during the flu season (September through February). \\
		\hline
		Diabetes: Foot Exam & The percentage of patients aged 18-75 years with diabetes (type 1 or type 2) who had a foot exam (visual inspection, sensory exam with monofilament, or pulse exam). \\
		\hline
		Appropriate Testing for Children with Pharyngitis & Percentage of children 2-18 years of age, who were diagnosed with pharyngitis, dispensed an antibiotic and received a group A streptococcus (strep) test for the episode. \\
		
		\hline
		\end{tabular}
		\label{table:cqm}
		\caption[Table caption text]{Example clinical quality measures taken from \cite{cqm-pdf}}
\end{table}


Providers participating in the Meaningful Use incentive program must report annually on at least a subset of Clinical Quality Measures as defined by CMS.
These measures cover these priorities for health care quality improvement as established by HHS \cite{cqm-intro}:
\begin{enumerate}
	\item Patient and Family Engagement
	\item Patient Safety
	\item Care Coordination
	\item Population/Public Health
	\item Efficient Use of Healthcare Resources
	\item Clinical Process/Effectiveness
\end{enumerate}

Based on my review of the 2014 clinical quality measures for outpatient care as in \cite{cqm-pdf}, the measures typically affirm that the medical practice is providing what should be standard maintenance care for chronic conditions with a goal of improving health and reducing long term costs. I show some examples in table \ref{table:cqm}.

\section{Introducing Software as a Service (Saas)}
\label{sec:Intro_SaaS}

As mentioned in section \ref{sec:Introduction} a cloud-based delivery model for EHR systems is likely to be appealing for medical providers since these systems typically require much less up front investment and should offload maintenance and overhead to the vendor of the system.
The cloud-based delivery of software is typically referred to as Software as a Service (SaaS).
% TODO -- provide a better intro of what SaaS is here.

SaaS differs from traditional software products in that it uses a multi-tenant architecture.
In other words, customers share the same codebase and physical infrastructure in SaaS versus the traditional software model where each customer is using a distinct installation and distinct infrastructure.
An advantage to this configuration is that customers can expect higher network bandwidth, reliability, and relatively lower cost.
However, this model does constrain the ability to customize the software for different customers.
Furthermore, vendors are typically in charge of when and how software upgrades are rolled out meaning customers generally have no choice but to adopt upgrades \cite{saasqual}.

\subsection{Advantages of SaaS for EHR}
\label{sec:SaaS Adantages}

In this section I will cover the typical advantages of SaaS software for customers and how these specifically apply to an EHR-SaaS system.

\begin{description}
\item[Cost]
SaaS products are usually accessed by users via an internet client directed to a specific website.
In this model, the users typically pay a regular licensing or subscription fee and do not have to make any up front purchase or hardware investments other than to acquire client machines (desktops, laptops, and mobile devices). 
A doctor's office will typically need to acquire client machines in order to conduct business. Overall, choosing a SaaS will require less expensive and less total hardware than a more traditional EHR system.

\item[Reliability]
Practices can negotiate contracts with vendors with specific uptime guarantees \cite{wiki-saas}, problem resolution and customer service response.
My estimate is that a vendor should be able to provide much more rigorous guarantees for these concerns than a typical in-house IT staff.

\item[Maintenance and upgrades]
Software customers often dread (with good reason) the moments when they are required to update the software or hardware they have been using.
Since a SaaS product is usually hosted by the vendor and delivered to the clients via a web browser, then vendor then takes on the cost and responsibility of rolling out software updates.
Furthermore, the vendors of a SaaS product manage their own data centers and server hardware completely out of sight of customers.
Most SaaS providers will deliver software updates to users automatically and much more frequently than is typical of a traditional software product.
Frequent updates offer many advantages to the end user including more rapid delivery of new features and more incremental changes that reduce the likelihood of major outages or breakages and reduce the scope of changes that may confuse the user.
For a medical practice, this advantage will be a bit of a mixed blessing.
On one hand, the practice will no longer bear the burden of undertaking a software upgrade.
Furthermore, the practice will only need to maintain local client machines and infrastructure and will not need to concern itself at all with server and other hard centralized hardware upgrades.
On the other hand, the practice will have little to no choice about when to take an upgrade and overall a medical practice is more interested in stability of their software than the average cloud service customer.

\item[Security]
Since the vendor is hosting the data, data security is an area of concern that is typically listed as a disadvantage of SaaS \cite{wiki-saas}.
However, in the case of EHR systems, I would argue that the software vendor is in a better position to ensure data security as they should have greater expertise in IT security than a typical medical provider office.
Furthermore, the fines enacted by HIPAA against negligence around and misuse of PHI motivate anyone dealing with this type of information to take appropriate precautions.

\end{description}

\subsection{Challenges of SaaS for EHR}
\label{sec:SaaS Challenges}

In this section, I will discuss the aspects of a SaaS software product that are likely to be a challenge and/or a concern from the EHR point of view.

\begin{description}
\item[Customization]
The very thing that drives the cost-efficiency of a SaaS product, i.e. multi-tenancy, limits the ability of the product to be customized for a specific customer.
This concern is particularly strong when it comes to medical practice workflows, i.e. the processes a medical office typically follows for a patient visit, for scheduling, to process referrals, and so on.
There is a saying in the medical field: "When you've seen one medical practice, you've seen one medical practice" that strongly reinforces this concern \cite{health-hack}.
It may be possible to convince medical practices in the same specialty to converge to a more or less similar workflow even though it will involve retraining nearly every member of the provider's staff, including the doctors. 
However, it will be nearly impossible to expect practices from disparate specialties such as pediatrics vs. oncology to converge to the same workflows.
Therefore, EHR software will need to be customized at least for each specialty.

\item[Access and transferability of data]
With SaaS-EHR systems, individual patient data and other patient and practice specific data will be hosted on the vendor's servers.
If the medical provider ever wishes to change EHR systems, her office will need to be able to transfer records from one vendor to the other.
The provider should take care to address exit strategies and concerns with any vendor offering an EHR system \cite{ehr-breakup}.
Even with the most well though out contract and plans, the transition from one EHR system to another will be a significant disruption to the practice.

\item[Stability]
SaaS providers update their versions frequently and typically force users to use the latest version of the software.
While this is often a good thing, some users may be frustrated by changes in product they have become used to \cite{wiki-saas}.
In EHR, this is a much greater concern since these changes may result in errors made by the users (the medical provider's staff) that impact patient safety in a negative way.
See the discussion of "Maintenance and upgrades" in \ref{sec:SaaS Adantages}.

\end{description}

Overall, an EHR system delivered as a service should be very attractive to medical providers whose main mission is to provide high quality medical care and not to become IT and Health IT organizations.
However, adopting practices need to seriously consider and plan for the above challenges.

\section{Differentiating an EHR-SaaS Product}
\label{sec:Critical-EhrSaaS-Ilities}

As I mentioned in section \ref{sec:Introduction}, the functional requirements (what the system must do) for an EHR-SaaS product are largely mandated by the certification requirements set by the ONC.
Therefore, the area where a certified EHR-SaaS product can differentiate itself are in the non-functional requirements (how the system must be) or the "qualities" of the system \cite{wiki-nfr}.
Examples of non-functional requirements are usability, security, privacy, reliability, and so on.
These are sometimes even referred to as "ilities".
No software product can focus on each and every possible quality and achieve satisfactory time to market and a competitive price point.
Therefore, developers of a software product must focus on the qualities that customers consider most important and make the best possible compromise when important qualities are contradictory.


\subsection{Critical Non-Functional Requirements of a SaaS Product}
\label{sec:Critical-SaaS-Ilities}

Over the past couple of decades, researchers have engaged in extensive study on what qualities customers consider important for traditional (non-SaaS) software products.
However, the research on traditional products is insufficient for SaaS products as too many aspects of the delivery, support, installation, maintenance, pricing and so on are different.
Benlian, Koufaris, and Hess (2011, \cite{saasqual} engaged in a rigorous study in \cite{saasqual} of the qualities a purchaser would consider most important in deciding whether or not to continue with a SaaS product.
They determined the below factors as having primary importance in this order with responsiveness and security as clear leaders:
\begin{description}
	\item[Responsiveness]
	Consists of all aspects of a SaaS provider’s ability to ensure that the availability and performance of the SaaS-delivered application (e.g., through professional disaster recovery planning or load balancing) as well as the responsiveness of support staff (e.g., 24-7 hotline support availability) is guaranteed.
	\item[Security]
	Includes all aspects to ensure that regular (preventive) measures (e.g., regular security audits, usage of encryption, or antivirus technology) are taken to avoid unintentional data breaches or corruptions (e.g., through loss, theft, or intrusions).
	\item[Flexibility]
	Covers the degrees of freedom customers have to change contractual (e.g., cancellation period, payment model) or functional/technical (e.g., scalability, interoperability, or modularity of the application) aspects in the relationship with a SaaS vendor.
	\item[Rapport]
	Includes all aspects of a SaaS provider’s ability to provide knowledgeable, caring, and courteous support (e.g., joint problem solving or aligned working styles) as well as individualized attention (e.g., support tailored to individual needs).
	\item[Reliability]
	Comprises all features of a SaaS vendor’s ability to perform the promised services timely, dependably, and accurately (e.g., providing services at the promised time, provision of error-free services).
	\item[Features]
	Refers to the degree the key functionalities (e.g., data extraction, reporting, or configuration features) and design features (e.g., user interface) of a SaaS application meet the business requirements of a customer.
\end{description}
(All definitions above are quoted directly from \cite{saasqual}).

Fortunately, Benlian, et al.'s study was specifically on business to business (B2B) SaaS products. 
EHR systems do belong to the B2B category although as a very unique and distinct type of B2B sotware.
I consider the study and paper by Benlian, et. al in \cite{saasqual} to be excellent and rigorous.
This paper provides both an exemplary model for doing similar research regarding EHR-SaaS products and a great starting point for discussion of which qualities should be prioritized for an EHR-SaaS product.

 % TODO - benlian also go into 'zones of tolerance' which imply/use a scale which might be useful when/if I get into actual measurements in this paper

\subsection{Conjectures on Critical Non-Functional Requirements of an EHR-SaaS Product}
\label{sec:Critical-EHR-Ilities}

Unfortunately, I was not able to find papers studying the criticality of non-functional requirements for EHR-SaaS systems with the same rigor as Benlian, et al. in \cite{saasqual}.
However, from their research, I can start outlining thoughts about what might be unique about an EHR-SaaS product versus a generic SaaS product.
% err, offer a critique

First, in \cite{saasqual}, the authors are primarily concerned with the continuance of the usage of a SaaS product and not on how quality factors influence the decision to adopt a SaaS product.
I suppose that they focus on continuance rather than adoption because, as they state it, customers can switch SaaS providers easily \cite{saasqual}. 
I am not convinced that it is that easy for a customer to switch between SaaS providers even in certain fairly mundane cases.
However, in the case of EHR, a medical practice will definitely find it very expensive and time consuming to switch EHR providers due, at minimum, to the sheer amount of data that must be ported and to the amount of time and expense required to train the medical staff.
Therefore, in further studies on EHR-SaaS, I would recommend analysis of both qualities important to win new customers and qualities important to retain (happily) existing customers.
For the sake of this discussion, it is important to acknowledge that the focus of the authors of \cite{saasqual} is too much on customer retention and not enough on customer acquisition for the EHR-SaaS purpose due to their assumption about the ease of switching vendors.


In \cite{saasqual}, the authors assert that the primary reasons customers discontinue a SaaS product are unfulfilled technical requirements, security issues and bad customer support.
Given that security around EHR data is strongly mandated by HIPAA, which enforces strong civil and criminal penalties, I suppose that nearly all EHR-SaaS solutions should have very good security and privacy controls.
Therefore, security is unlikely to be a top reason to switch EHR systems as an EHR system with security shortcomings would have bigger problems than merely losing a customer or two.
Similarly, the technical requirements of an EHR system are to a large extent mandated by the ONC.
Therefore, it is highly likely that most EHR systems will have the technical functionality required.
However, bad customer support is likely to remain a good reason to switch EHR-SaaS vendors, which is reinforced by Brad Benson in \cite{switch-ehr}.
Therefore, I would reduce in importance "Security" and "Features" for the purpose of retaining or acquiring EHR-SaaS customers but I would retain "Rapport" and "Responsiveness" as very important for an EHR-SaaS system.
Please note that I do not mean "Security" or "Features" is unimportant; 
I simply propose that they are less important non-functional requirements for the purpose of differentiating an EHR-SaaS product than other non-functional requirements.

Benson's article also mentions "EHR design not suited for the practice specialty or specialties" as a top reason to switch vendors.
As I mentioned in \ref{sec:SaaS Challenges}, the workflow in each medical practice is essentially unique.
Even if a vendor can convince all of its cardiology practices to adopt workflows similar enough to use the same software product, the vendor will in all likelihood need to provide customizations between specialties at the very least.
Therefore, I would include "Customization" or "Customizability" as a critical quality for EHR-SaaS products.

In section \ref{sec:Goals of an EHR}, I discussed the non-functional requirements for an EHR system (regardless of implementation type) that I encountered in the literature.
In this section, to summarize, portability, interoperability, security and privacy, and reliability were listed.
As mentioned previously, security and privacy are mandated by HIPAA, so I will not include them in a list of differentiating features.
I will, however, retain reliability as a key quality feature from the work by Benlian.
A given practice will need to cooperate with countless other medical practices in the course of caring for patients.
Given the freedom of choice of EHR providers, these other practices will often and most likely have different implementations of an EHR system.
Therefore, I will add interoperability as a key differentiating feature, and include portability under that heading.

In section \ref{sec:Goals of an EHR}, I mentioned one challenge of SaaS adoption for EHR and for other systems is stability.
Namely, the SaaS delivery model puts the vendor in control of when and how new versions of software is rolled out.
Most SaaS vendors roll out relatively smaller updates frequently to customers, and customers have no choice but to use the latest version.
Often, users barely notice that their software has been updated and proceed happily along.
However, often, changes to the way a frequently used software product is disturbing to users.
For example, a reader familiar with Facebook may have noticed how angry some users have become when Facebook rolls out updates to how the News Feed works.
In the case of EHR-SaaS, a vendor must take particular care with rolling out updates to users particularly those that meaningfully change how things work.
Users will need retraining and to relearn a system if things change too much.
In the worst case, users could end up making a mistake regarding patient data and encounters;
such a mistake could very well lead to patient safety issues.
Therefore, I would list stability as being a very critical quality factor for EHR-SaaS systems.

Finally, I would retain flexibility as an important quality factor for EHR-SaaS systems as practices may wish to vary how they pay for a system or even which parts of a larger EHR system they choose to use.
For example, a pharmacy will wish to participate in EHR, but it should only need to receive prescriptions from practices and not typically manage patient encounters or send orders.

Here is my nonprioritized list of quality factors I conjecture are most critical for an EHR-SaaS system.
\begin{itemize}
	\item Rapport (affirmed) 
	\item Responsiveness (affirmed)
	\item Customization (added)
	\item Reliability (affirmed)
	\item Stability (added)
	\item Flexibility (affirmed)
	\item Interoperability (added)
\end{itemize}

I would encourage a similar study as rigorous as Benlian, et al.'s take place on important quality factors for the adoption and continuance of SaaS in order to validate this list, introduce or remove factors, and prioritize the importance of quality factors.

\section{Conclusions and Future Work}

Conclusions go here. 

\bibliographystyle{abbrv}
\bibliography{refs}
\end{document}
